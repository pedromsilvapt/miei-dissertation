%!TEX root = ../template.tex
%%%%%%%%%%%%%%%%%%%%%%%%%%%%%%%%%%%%%%%%%%%%%%%%%%%%%%%%%%%%%%%%%%%%
%% acknowledgements.tex
%% NOVA thesis document file
%%
%% Text with acknowledgements
%%%%%%%%%%%%%%%%%%%%%%%%%%%%%%%%%%%%%%%%%%%%%%%%%%%%%%%%%%%%%%%%%%%%
\acknowledgements
Quando estamos perante um momento importante da nossa vida, é fácil pensarmos nele como um momento circunspecto e isolado. Uma fase que acaba e outra que começa. Um antes e um depois. Mas a minha experiência até agora mostrou-me que na verdade, a minha vida tem sido mais um contínuo. E nesse contínuo estão um conjunto de pessoas sem o qual seria impossível pensar sonhar em fazer o que fiz.

Por isso mesmo, em primeiro lugar, gostava de agradecer profundamente ao meu orientador de dissertação, o Professor José João Almeida, pela ajuda durante todo este processo. Sempre pronto a transmitir experiência e conhecimento, novas ideias, críticas construtivas e positivas, incentivando-me a agarrar todas as oportunidades, mas mais importante que tudo, encorajando-me sempre a seguir ao meu próprio ritmo e tomar o meu próprio caminho.

Gostava também de agradecer aos meus colegas de curso, em especial à Sofia Silva, sempre pronta a ajudar para qualquer dúvida que eu tivesse e para me dar motivação para seguir em frente.

Também não me posso esquecer de agradecer a todos os professores que tive, em particular aos professores Carlos Carvalho, Luís Cerejeira, Daniel Rego e Pedro Ribeiro, que me ensinaram, inspiraram, e se esforçaram para me providênciaram oportunidades e experiências que nunca irei esquecer, e que me abriram portas na vida que de outra forma me estariam fechadas.

E mais importante que tudo, gostava de agradecer à minha família. À minha mãe, que sempre me apoiou durante todo o curso e fez o que fosse preciso para tornar a minha experiência o melhor possível. À minha tia, uma das pessoas mais trabalhadoras que conheço, que não pensa duas vezes antes de me ajudar em tudo o que eu precisar, e que está sempre presente para me apoiar. E finalmente ao meu irmão, a quem eu sempre quis seguir os passos, e que me fez querer começar a programar desde os doze anos de idade. Se não fosse por ele, não estaria a fazer esta dissertação, nem a acabar este curso, nem teria tido as oportunidades que tive.

A todos, estou eternamente grato. Muito obrigado.
