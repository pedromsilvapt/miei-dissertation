%!TEX root = ../template.tex
%%%%%%%%%%%%%%%%%%%%%%%%%%%%%%%%%%%%%%%%%%%%%%%%%%%%%%%%%%%%%%%%%%%%
%% abstrac-en.tex
%% NOVA thesis document file
%%
%% Abstract in English
%%%%%%%%%%%%%%%%%%%%%%%%%%%%%%%%%%%%%%%%%%%%%%%%%%%%%%%%%%%%%%%%%%%%
\textbf{Title:} Musikla - Music and Keyboard Language

\bigskip

The creation of music using synthesizers is a practice that boasts many decades of existence. With the proliferation of personal computers, digital synthesizers and Digital Audio Workstations rose in popularity as well.

This project aims to use the digital audio production and manipulation technologies, and wrap them in a domain specific language that allows to easily describe music compositions, generate sounds dynamically, study properties of music theory or even create virtual keyboards that play sounds or execute actions and can be used to perform live multimedia shows.
% Palavras-chave do resumo em Inglês
\begin{keywords}
Domain Specific Language, Interpreter, Language Processing, Music Notation \ldots
\end{keywords} 
