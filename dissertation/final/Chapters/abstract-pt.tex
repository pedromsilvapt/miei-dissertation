%!TEX root = ../template.tex
%%%%%%%%%%%%%%%%%%%%%%%%%%%%%%%%%%%%%%%%%%%%%%%%%%%%%%%%%%%%%%%%%%%%
%% abstrac-pt.tex
%% NOVA thesis document file
%%
%% Abstract in Portuguese
%%%%%%%%%%%%%%%%%%%%%%%%%%%%%%%%%%%%%%%%%%%%%%%%%%%%%%%%%%%%%%%%%%%%
A criação de música usando sintetizadores é uma prática que conta já com várias décadas de uso. Com a proliferação dos computadores pessoais, popularizaram-se também os sintetizadores digitais e as Digital Audio Workstations.

Este projeto tem como objetivo utilizar as tecnologias de produção e manipualação de aúdio digital, e envolvê-las numa linguagem de domínio específico que permita facilmente descrever composições músicais, gerar sons dinâmicamente, estudar propriedades da própria teoria músical, ou até criar teclados virtuais que toquem sons ou executem ações e podem ser utilizados para realizar espetáculos multimédia ao vivo.
% Palavras-chave do resumo em Português
\begin{keywords}
Palavras-chave (em Português) \ldots
\end{keywords}
% to add an extra black line
