%!TEX root = ../template.tex
%%%%%%%%%%%%%%%%%%%%%%%%%%%%%%%%%%%%%%%%%%%%%%%%%%%%%%%%%%%%%%%%%%%
%% chapter1.tex
%% NOVA thesis document file
%%
%% Chapter with introduciton
%%%%%%%%%%%%%%%%%%%%%%%%%%%%%%%%%%%%%%%%%%%%%%%%%%%%%%%%%%%%%%%%%%%
\newcommand{\novathesis}{\emph{novathesis}}
\newcommand{\novathesisclass}{\texttt{novathesis.cls}}


\chapter{Introdução}
\label{cha:introducao}

%\epigraph{
  %This work is licensed under the Creative Commons Attribution-NonCommercial~4.0 %International License.
  %To view a copy of this license, visit \url{http://creativecommons.org/licenses/%by-nc/4.0/}.
%}

% Context,\\ motivation,\\ main aims	(objectives) \\ research hypothesis, (optional) \\ paper organization!
No âmbito das linguagens de programação e engenharia de \textit{software}, é comum observar como a linguagem, ou o paradigma, de programação escolhido pode afetar de modo profundo as decisões que o programador toma, e por consequência, o produto final. De modo análogo, nós acreditamos que a forma como pensamos e falamos sobre música afeta também a música que criamos, ou a forma como pensamos sobre ela. A linguagem musical mais comum utilizada entre profissionais e entusiastas é certamente a linguagem das pautas musicais.

Mas essa linguagem é estática, na medida em que representa apenas o resultado final. Podemos compará-la a falar de matemática usando apenas números, mas sem nenhuma linguagem comum para expressarmos conceitos de mais alto nível, tais como equações ou inequações. Era certamente possível representarmos os resultados finais, mas perdia-se a estrutura que possibilitava compreender a um mais alto nível o problema, e até mesmo generalizar a sua solução e encontrar padrões para outros problemas similares.

Sendo o processamento de linguagens, em particular de linguagens de programação, uma área de enorme interesse pessoal, surgiu a ideia de como integrar vários desses conceitos na linguagem musical, e desta forna torná-la mais rica e poderosa.

O objetivo desta dissertação irá então passar por desenhar e desenvolver uma linguagem que pode ser pensada como uma calculadora musical. Queremos analisar qual a melhor forma de descrever música em formato textual, e de integrar construções comuns na programação, como variáveis, funções, ciclos e estruturas de decisão, entre outros.

Esta linguagem deve então servir tanto como uma ferramenta de suporte teórico, para os utilizadores estudarem a estrutura e a teoria de peças musicais, mas também como uma ferramenta prática para praticar a criação de música.

Esta última parte prática motivou também a incorporação na nossa linguagem de uma componente de interatividade. Para além de permitir usar pianos, ou o próprio teclado de um computador, para reproduzir as notas habituais, queremos dar ao utilizador a possibilidade de personalizar as ações ou os acompanhamentos musicais que tocam em cada tecla.

Um dos objetivos que iremos repetir bastante ao longo desta dissertação é a necessidade deste projeto ser extensível. Isto é, permitir a cada utilizador adicionar novas funcionalidades à linguagem quando precisar, e tornar este processo o mais simples possível, sem fases de recompilação ou necessidade de alterar o código fonte do projeto. Isto porque a produção de música é uma área enorme, e mais importante, é uma forma de expressão criativa. Não temos ilusões de conseguir cobrir sequer a maior parte dos casos de uso de todos os utilizadores. E por isso, providenciar uma base sólida sobre a qual cada pessoa pode facilmente construir, e eventualmente ir juntando essas construções à implementação oficial, parece-nos crítico para o sucesso e para a usabilidade da linguagem.

Para realizar isso, iremos analisar no próximo capítulo as linguagens existentes na interseção da programação com a criação de música. Vamos analisar para cada uma, quais os aspetos que consideramos estarem em falta relativamente ao que queremos encontrar numa linguagem, mas também aquilo que já existe e que está bem feito nessas linguagens. E como as boas ideias são para ser aproveitadas, iremos incorporar o que for sensato na nossa linguagem (como por exemplo, usar a sintaxe de descrição de notas e acordes do projeto \textit{abc notation}\footnote{\url{http://abcnotation.com/}}).


%O objetivo deste trabalho é estudar e prototipar formas de criação de música com recurso a técnicas habitualmente usadas na criação de \textit{software}. Para além de permitir criar música através das notas introduzidas manualmente, a linguagem deve facilitar a geração de música de um modo mais dinâmico, com recurso a operações de combinação e transformação de notas, como concatenar e misturar música.

%O termo música neste contexto é usado num sentido mais amplo que apenas sons. O objetivo desta linguagem é permitir gerar vários \textit{outputs} através do mesmo código fonte, como pautas músicais, ABC, WAV, MIDI, entre outros.

%Uma das partes mais críticas relativas à pesquisa e desenvolvimento necessários para a realização desta linguagem é a componente temporal implícita em todos os aspetos da linguagem: deve ser possível de um modo intuítivo expressar as várias composições possíveis de notas sem ser necessário expressar os tempos manualmente, tais como notas sequênciais, notas em paralelo, acordes, pequenas pausas e grandes pausas, bem como sincronizar partes da música de modo simples.
