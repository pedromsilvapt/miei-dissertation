%!TEX root = ../template.tex
%%%%%%%%%%%%%%%%%%%%%%%%%%%%%%%%%%%%%%%%%%%%%%%%%%%%%%%%%%%%%%%%%%%%
%% chapter4.tex
%% NOVA thesis document file
%%
%% Chapter with lots of dummy text
%%%%%%%%%%%%%%%%%%%%%%%%%%%%%%%%%%%%%%%%%%%%%%%%%%%%%%%%%%%%%%%%%%%%
% CHAPTER - Conclusion/Future Work --------------
\chapter{Conclusão}
O desenvolvimento do projeto envolve vários aspetos, desde o desenho da sintaxe e da gramática correspondente, até à geração de sons a serem guardados em ficheiros ou reproduzidos imediatamente em dispositivos aúdio. Também engloba aspetos comuns em todoas as linguagens de programação, em conjunto com questões mais específicas sobre como gerir o conceito de tempo na linguagem, de \textit{laziness} para gerar apenas as notas necessárias, entre muitos outros.

Mas para além disso também abrange quais as ferramentas e as metodologias que são mais adequadas para a utilização da linguagem. É necessário para isso estudar com exemplos reais, qual a melhor forma de produzir e desenvolver música em formato textual. Quais as funções e as suas interfaces que são mais úteis disponibilizar logo à partida a todos os útilizadores.

No entanto, sendo a criação músical uma área tão ampla, é ínutil tentar sequer conseguir desenvolver uma ferramenta que cubra todos os cantos e sirva todas as necessidades dos seus potênciais utilizadores. É por isso que é importante apoiar o desenvolvimento do projeto nas vantagens que a linguagem \textit{Python} fornece, quer a nível da sua facilidade de uso, popularidade, e fácil extensão sem necessidade de complicados processos de compilação.

O projeto deve ser por isso desenvolvido com extensibilidade em mente, tendo como objetivo principal servir como uma fundação estável capaz de ligar as diversas ferramentas existentes, não só na área da música, mas permitir ligar a música a outras áreas.

Durante o desenvolvimento, tornou-se claro que criar música através de programação é uma forma extremamente poderosa de se trabalhar em muitas situações, mas não é uma solução perfeita para todos os casos. Com isto em mente, sentimos que a nossa escolha de incluir diretamente na nossa linguagem a possibilidade de programar teclados permite cobrir os momentos em que o utilizador precisa de uma forma mais interativa e imediata de tocar e compôr músicas. Para além disso, as funcionalidades extra de permitir gravar e reproduzir as \textit{performances} em \textit{buffers} enquanto os teclados estão ativos, bem como a inclusão de um simples editor de código que permite escrever e executar código em \textit{runtime} ofereceram uma experiência de composição e experimentação com ágil e rápido \text{feedback loop}.

A nível de processamento de linguagem, sentimos que a escolha de usar um interpretador \textit{tree-walk} favoreceu muito a prototipagem de funcionalidades na linguagem. Mas fica sempre aberta a possibilidade de implementação da linguagem com um \textit{backend} alternativo, quer através de uma maquina virtual de \textit{bytecode}, ou possivelmente até \textit{convertendo} o código Musikla para código \textit{Python} e usando as funcionalidades de execução em \textit{runtime} do mesmo.

A coleção de métodos e classes disponibilizados pela linguagem, e mais especificamente, pela biblioteca \textit{standard} desenvolvida para acompanhar a linguagem, já permite o desenvolvimento de \textit{scripts} bastante complexos e interessantes, como pode ser visto em muitos dos casos de estudo. Ainda assim, é indiscutível que existem ainda uma enorme quantidade de métodos e primitivas ligadas à teoria músical ou à geração procedural de música por adicionar, pois tal é uma tarefa sempre em evolução.

Ficamos também muitos satisfeitos por reutilizar padrões, técnicas e bibliotecas onde existem já soluções bem estabelecidas, tanto a nível de formatos de \textit{input/output}, como o caso do formato \textit{MIDI} ou \textit{abc}, que nos permitem apanhar boleia de todo o ecossistema de ferramentas construído já à sua volta. Também a utilização de módulos como o FluitSynth, que nos permitiu não só fazer programas que escrevessem ou returnassem notação musical como resultado, mas tocassem as notas em \textit{realtime}, com suporte para vários instrumentos, sem o utilizador necessitar de configurar nada externo à aplicação, tornou-a a nosso ver muito mais interessante. Por fim, a utilização de notação existente (como a do \textit{abc notation}) dentro da nossa própria linguagem reduz imenso a curva de aprendizagem necessária para alguém (quer já seja músico ou não) se tornar proeficiente com ela.

Tudo isto resulta, na nossa opinião, num projeto que não reinventa a roda, mas sim adota os standards existentes onde estes fazem sentido, e pode assim focar-se em colmatar as áreas que realmente faltavam no ecossistema: uma linguagem declarativa e dinâmica para descrição de acompanhamentos músicais, e da possibilidade de criação de teclados interativos que combinam sem fricção com o resto da linguagem. 

E assim esperamos que este projeto sirva como uma fundação. Que permita a mais pessoas no futuro fazer como nós fizemos, aproveitando as ideias que já existam e encontrem novas aberturas na área da música computacional. E desta vez aproveitando boleia da nossa linguagem, como nós fizemos também, criem novos projetos, novas ideias e novas possibilidades.
