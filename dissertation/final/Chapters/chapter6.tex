%!TEX root = ../template.tex
%%%%%%%%%%%%%%%%%%%%%%%%%%%%%%%%%%%%%%%%%%%%%%%%%%%%%%%%%%%%%%%%%%%%
%% chapter4.tex
%% NOVA thesis document file
%%
%% Chapter with lots of dummy text
%%%%%%%%%%%%%%%%%%%%%%%%%%%%%%%%%%%%%%%%%%%%%%%%%%%%%%%%%%%%%%%%%%%%
% CHAPTER - Conclusion/Future Work --------------
\chapter{Conclusão}
O desenvolvimento do projeto envolve vários aspetos, desde o desenho da sintaxe e da gramática correspondente, até à geração de sons a serem guardados em ficheiros ou reproduzidos imediatamente em dispositivos aúdio. Também engloba tanto aspetos mais genéricos da programação, como qual a melhor forma de implementar de forma opaca funções geradoras e funções assíncronas.

Mas para além disso também abrange quais as ferramentas e as metodologias são mais adequadas para a utilização da linguagem. É necessário para isso estudar com exemplos reais, qual a melhor forma de produzir e desenvolver música em formato textual.

No entanto, sendo a criação músical uma área tão ampla, é ínutil tentar sequer conseguir desenvolver uma ferramenta que cubra todos os cantos e sirva todas as necessidades dos seus utilizadores. É por isso que é importante apoiar o desenvolvimento do projeto nas vantagens que a linguage \textit{Python} fornece, quer a nível da sua facilidade de uso, popularidade, e fácil extensão sem necessidade de processos complicados de compilação.

O projeto deve ser por isso desenvolvido com extensibilidade em mente, tendo como objetivo principal servir como uma fundação estável capaz de ligar as diversas ferramentas existentes, não só na área da música, mas permitir ligar a música a outras áreas.
