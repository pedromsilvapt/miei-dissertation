%!TEX root = ../template.tex
%%%%%%%%%%%%%%%%%%%%%%%%%%%%%%%%%%%%%%%%%%%%%%%%%%%%%%%%%%%%%%%%%%%%
%% glossary.tex
%% NOVA thesis document file
%%
%% Glossary definition
%%%%%%%%%%%%%%%%%%%%%%%%%%%%%%%%%%%%%%%%%%%%%%%%%%%%%%%%%%%%%%%%%%%%

\newglossaryentry{acidental} {
	name={acidental}, 
	description={Corresponde à nota de uma frequência que não pertence à escala músical atual. Representam-se pelos símbolos sustenido $\sharp$ (\textit{sharp} em inglês), bemol $\flat$ (\textit{flat}) ou bequadro $\natural$ (\textit{natural}), e que geralmente aumentam ou diminuem a frequência da nota por um semitom.}
}

\newglossaryentry{acorde} {
	name={acorde}, 
	description={Conjunto de várias notas que são geralemente tocadas em simultâneo.}
}

\newglossaryentry{arpeggio} {
	name={arpeggio}, 
	description={Passa pela reprodução sequêncial das notas de um acorde (em vez de as tocar em simultâneo). Pode ser simples (tocar cada nota uma vez em sequência) ou seguir uma melodia mais complexa usando as notas disponíveis.}
}

\newglossaryentry{escala diatônica} {
	name={escala diatônica}, 
	description={Uma escala músical composta por sete notas, como um intervalo de 12 semitons entre as suas notas. Este padrão de notas repete-se a cada oitava nota, subindo ou descendo a sua frequência.}
}

\newglossaryentry{ligaduras} {
	name={ligaduras}, 
	description={Símbolo colocado nas partituras músicais para ligar notas seguidas com o mesmo \textit{pitch} em diferentes compassos. As durações das notas são somadas e elas são tocadas como uma só nota.}
}

\newglossaryentry{oitava} {
	name={oitava}, 
	description={Intervalo entre uma nota músical, e a correspondente com  dobro ou metade da sua frequência. Na escala diatônica, uma oitava corresponde a 12 semitons.}
}

\newglossaryentry{semitom} {
	name={semitom}, 
	description={Também chamado de meio-tom, é o menor intervalo utilizado na escala diatônica. Representa a distância tónica entre duas teclas adjacentes de um piano.}
}



\newglossaryentry{sintetizador} {
	name={sintetizador}, 
	description={Dispositivo ou \textit{software} responsável por gerar sinais de aúdio em tempo real.}
}

\glsaddall
