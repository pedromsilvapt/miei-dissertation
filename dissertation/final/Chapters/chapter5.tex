%!TEX root = ../template.tex
%%%%%%%%%%%%%%%%%%%%%%%%%%%%%%%%%%%%%%%%%%%%%%%%%%%%%%%%%%%%%%%%%%%%
%% chapter4.tex
%% NOVA thesis document file
%%
%% Chapter with lots of dummy text
%%%%%%%%%%%%%%%%%%%%%%%%%%%%%%%%%%%%%%%%%%%%%%%%%%%%%%%%%%%%%%%%%%%%
\chapter{Desenvolvimento}
O desenvolvimento do projeto pode ser dividido de grosso modo em três. Nessas três camadas são cobertos um grupo abrangente de aspetos tanto da área do processamento de linguagens e do desenvolvimento de \acrshort{dsl}'s, da teoria músical, e do processamento digital de audio. 

Na camada da linguagem esteve mais proeminente a área de processamento de linguagens, por motivos óbvios. Mas nas decisões tomadas durante o desenvolvimento desta camada, estiveram sempre presentes também as necessidades específicas que a teoria músical (e a sua notação) impõeem numa linguagem de programação.

Do mesmo modo, o interpretador faz claramente uso de tópicos do domínio do processamento de linguagens, mas é ainda mais fortemente influencidado pelas restrições  e requisitos impostos pela componente músical da linguagem. Esta influencia a forma e a semântica da execução dos vários operadores disponibilizados.

A última camada, de desenvolvimento de uma biblioteca, composta pelos objetos e procedimentos que têm como objetivo facilitar a utilização da linguagem. Para isso foi necessário identificar os casos de utilização mais comuns e prioritários, de modo a guiar a construção destas interfaces para refletirem uma utilização real da aplicação.

\section{Linguagem}
A camada sintática da aplicação pode ser conceptualmente dividida em duas fases:
\begin{description}
    \item[Sintaxe] Esta fase caracteriza-se por delinear qual a sintaxe usada pela linguagem, bem como os construtores e operadores suportados;
    \item[Parser] Nesta fase foi desenvolvido um \textit{parser} em \textit{Python}, responsável por converter o código fonte da linguagem numa \acrfull{ast};
\end{description}

No entanto, a realidade é que a abordagem seguida (não só nesta camada mas como em todo o projeto) foi mais iterativa, dividindo cada faze em porções semi-independentes e intercalando as várias porções das diversas fases. Esta abordagem tem a vantagem de permitir ir testando e experimentando com o projeto mais cedo do que seria possível com um modelo de desenvolvimento em cascata.
\subsection{Sintaxe}
A sintaxe da linguagem é bastante inspirada nas usualmente chamadas linguagens da família C, com recurso a parênteses curvos e chavetas para delínear os vários blocos da linguagem. No entanto, as expressões são complementadas com um novo conjunto de literais e operadores dedicados a componente músical da linguagem. Conseguir juntar estes dois mundos trás consigo alguns desafios que serão discutidos mais em detalhe em cada umas das secções seguintes.

\subsubsection{\textbf{Literais}}
Literais referem-se ao conceito de sintaxe desenhada com o propósito de descrever data (literal) no código. São usados em quase todas as linguagens de programação (e na nossa também) para descrever números, \textit{strings} e booleanos.

A maior diferença nesta área entre a nossa linguagem foi a adição de literais responsáveis por modelar conceitos músicais, como notas, pausas e acordes. Esta sintaxe, tal como já foi mencionado anteriormente, foi inspirada pelo projecto \textit{abc notation}, com algumas modificações.

\subsubsection{\textbf{Notas e Pausas}}
A sintaxe de notas descrição de notas é composta por quatro componentes: \textbf{acidentais}, \textbf{\textit{pitch}} (obrigatório), \textbf{oitava} e \textbf{duração}.

\begin{center}
\begin{lstlisting}[backgroundcolor=\color{transparent},captionpos=b,caption=Expressão regular que identifica uma nota (quebras de linha apenas para claridade de leitura),xleftmargin=.4\textwidth]
[_^]*
[a-gA-G]
[',]*
([0-9]*\/)?[0-9]*
\end{lstlisting}
\end{center}

O \textit{pitch} refere-se à nota (ou frequência) que deve ser tocada. O \textbf{C médio} (também conhecido como C4) é descrito simplesmente como \texttt{C}. É possível descer uma ou mais oitavas acrescentando uma ou mais vírgulas \texttt{,}. Para subir uma oitava, podemos primeiro substituir as letras maíusculas por minúsculas. Para subir mais oitavas, podemos acrescentar uma ou mais pelicas \texttt{'}. Para subir ou descer semitons, podemos preceder a notas com os acidentais \texttt{\_} e \texttt{\^}.

As pausas são mais simples, sendo compostas simplesmente pela letra \texttt{r} seguida da sua \textbf{duração} (usando as mesma sintaxe das notas).

\subsubsection{\textbf{Acordes}}
Para definir acordes na linguagem, colocam-se várias notas dentro de parenteses retos. A notação usada para cada nota inclui os seus três primeiros componentes (acidentais, \textit{pitch} e oitava), mas exclui a duração. Em vez de definir a duração em cada nota, esta é definida globalmente no acorde após fechar os parenteses retos.

Por conveniência, para evitar ao utilizador ter de introduzir todas as notas de um acorde manualmente, temos uma sintaxe abreviada para os tipos de acordes mais comuns, onde é apenas necessário introduzir a nota base seguido do tipo de acorde.

\begin{table}[h]
\centering
\def\arraystretch{1.3}
\begin{tabular}{|l|c|}
\hline
\textbf{}        & \textbf{Abreviações}           \\ \hline
\textbf{Tríades} & M, m, aug, dim, +                \\ \hline
\textbf{Quinta}  & 5                                \\ \hline
\textbf{Sétimas} & m7, M7, dom7, 7, m7b5, dim7, mM7 \\ \hline
\end{tabular}
\caption{Lista de abreviaturas possíveis de serem acrescentadas a seguir a uma nota para especificar um acorde.}
\label{tab:modifiers-list}
\end{table}

A decisão de embrulhar cada acorde com parênteses retos deveu-se ao facto de muitas abreviaturas serem já populares no domínio da notação músical, e como tal o ideal era não as mudar. No entanto, algumas dessas abreviaturas poderiam entrar em conflito com outros componentes da declaração da nota. Por exemplo, \texttt{C7} poderia ser tanto um acorde de sétima como uma nota com duração de 7. Com a separação por parenteses retos, a ambiguidade deixa de existir, sendo obvio que \texttt{[C7]} é um acorde de sétima, e \texttt{C7} é uma nota com duração 7.

\begin{lstlisting}[caption={Exemplos de três definições de acordes possíveis}] 
[CFG]/4 [^Fm] [C5]2
\end{lstlisting}

\subsubsection{\textbf{Modificadores}}

Para além de permitir descrever notas, também é possível ter modificadores de contexto que permitem alterar certas propriedades das notas e acordes à sua frente, por exemplo.

\begin{table}[h]
\centering
\def\arraystretch{1.3}
\begin{tabular}{|l|c|c|c|}
\hline
\textbf{Nome} & \textbf{Modificador}                                            & \textbf{Exemplo}                                        & \textbf{Predefinição} \\ \hline
Instrumento   & I$N$                                                            & I46                                                     & I0           \\ \hline
Velocidade    & V$N$                                                            & V100                                                    & V127         \\ \hline
Tempo         & T$N$                                                            & T120                                                    & T60          \\ \hline
Duração       & \begin{tabular}[c]{@{}c@{}}L$N$\\ L/$N$\\ L$D$/$N$\end{tabular} & \begin{tabular}[c]{@{}c@{}}L2\\ L/4\\ L3/8\end{tabular} & L1           \\ \hline
Oitava        & O$N$                                                            & O2                                                      & O4           \\ \hline
Compasso      & S$D$/$N$                                                        & S3/4                                                    & S4/4         \\ \hline
\end{tabular}
\caption{Lista de modificadores e exemplos da sua utilização}
\label{tab:modifiers-list}
\end{table}

\subsubsection{\textbf{Operadores}}

\subsubsection{\textbf{Variáveis}}

\subsubsection{\textbf{Funções}}

\subsection{Parser}
\section{Interpretador}
\section{Biblioteca Standard}
\subsection{Inputs \& Outputs}
\subsection{Ficheiros de Som}
\subsection{Teclados Músicais}
\subsection{Grelhas}
\subsection{Transformadores}
\subsection{Editor Embutido}

