%% LyX 2.3.2 created this file.  For more info, see http://www.lyx.org/.
%% Do not edit unless you really know what you are doing.
\documentclass[table,xcdraw,english]{article}
\usepackage[T1]{fontenc}
\usepackage[utf8]{inputenc}
\usepackage{babel}
\usepackage{changepage}
\usepackage{enumitem}
\usepackage{url}
\usepackage[acronym]{glossaries}
\usepackage{tabu}
\usepackage{listings}
\usepackage[table]{xcolor}
\usepackage[toc,page]{appendix}

\renewcommand{\baselinestretch}{1.2}
\setlength{\parindent}{0pt}
\setlength{\parskip}{10pt}

\makeglossaries

\newacronym{dsl}{DSL}{Domain Specific Language}
\newacronym{pcm}{PCM}{Pulse-Code Modulation}
\newacronym{midi}{MIDI}{Music Instrument Digital Interface}
\newacronym{vm}{VM}{Virtual Machine}
\newacronym{mml}{MML}{Music Macro Language}
\newacronym{peg}{PEG}{Parsing Expression Grammar}
\newacronym{cfg}{CFG}{Context Free Grammar}

\begin{document}
\title{[WIP] Relatório de Pré-Dissertação Mestrado em Engenharia Informática}
\author{Pedro Miguel Oliveira da Silva}
\date{Setembro, 2019}
\maketitle

\section{Sinopse}
\begin{adjustwidth}{50pt}{50pt}
\begin{description}[labelwidth=70pt, leftmargin=!]
\item [{Candidato}] Pedro Miguel Oliveira da Silva
\item [{Tema}] DSL para programação de teclados e acompanhamentos musicais dinâmicos
virtuais
\item [{Orientação}] José João Almeida
\item [{Instituição}] Departamento de Informática \\ Escola de Engenharia \\ Universidade do Minho
\end{description}
\end{adjustwidth}


\newpage

\section{SoundFonts}
O formato \textit{SoundFont} foi originalmente desenvolvido nos anos 90 pela empresa E-mu Systems para ser usado inicialmente pelas placas de som Sound Blaster. Ao longo dos anos o formato sofreu diversas alterações, encontrando-se atualemente na versão 2.04, lançada em 2005\cite{soundfont}. Atualmente existem diversos sintetizadores de software \textit{cross platform} e {open source} capazes de converterem eventos \textit{MIDI} em som usando ficheiros SoundFont, dispensando a necessidade de uma placa de som compatível com o formato. Alguns destes projetos são TiMidity\verb|++|, WildMIDI e FluidSynth.

Um ficheiro de SoundFont é constítuido por um ou mais bancos (\textit{banks}) (até um máximo de 128). Cada banco pode por sua vez ter até 128 \textit{presets} (por vezes também chamados instrumentos ou programas). 


\textit{TODO}

\section{FluidSynth}
A biblioteca FluidSynth é um \textit{software} sintetizador de aúdio em tempo real que transforma dados MIDI em sons, que podem ser gravados em disco ou encaminhados diretamente para um \textit{output} de aúdio. Os sons são gerados com recurso a SoundFonts\cite{soundfont} (ficheiros com a extensão \texttt{.sf2}) que mapeiam cada nota para a gravação de um instrumento a tocar essa nota.

Os \textit{bindings} da biblioteca para C\verb|#| foram baseados no código \textit{open source} do projeto NFluidSynth\cite{nfluidsynth}, com algumas modificações para compilar com a versão da biblioteca em Linux.

\subsection{Inicialização}
Para utilizar a biblioteca FluidSynth, existem três objetos principais que devem ser criados: Settings (\texttt{fluid\_settings\_t*}), Synth (\texttt{fluid\_synth\_t*}) e AudioDriver (\texttt{fluid\_audio\_driver\_t*}).

O objecto \textbf{Settings}\cite{fluidsynth_settings} é implementado com recurso a um dicionário. Para cada chave (por exemplo, \texttt{``audio.driver''}) é possível associar um valor do tipo inteiro (\texttt{int}), \textit{string} (\texttt{str}) ou \textit{double} (\texttt{num}). Alguns valores podem ser também booleanos (\texttt{bool}), no entanto eles são armazenados como inteiros com os valores aceites sendo apenas 0 e 1.

O objeto \textbf{Synth} é utilizado para controlar o sintetizador e produzir os sons. Para isso é possível enviar as mensagens MIDI tais como \texttt{NoteOn}, \texttt{NoteOff}, \texttt{ProgramChange}, entre outros.

O terceiro objeto \textbf{AudioDriver} encaminha automaticamente os sons para algum \textit{audio output}, seja ele colunas no computador ou um ficheiro em disco. Os seguintes \textit{outputs} são suportados pela biblioteca:

\begin{adjustwidth}{20pt}{20pt}
\begin{description}
    \item[Linux:] jack, alsa, oss, PulseAudio, portaudio, sdl2, file
    \item[Windows:] jack, PulseAudio, dsound, portaudio, sdl2, file
    \item[Max OS:] jack, PulseAudio, coreaudio, portaudio, sndman, sdl2, file
    \item[Android:] opensles, oboe, file
\end{description}
\end{adjustwidth}

\subsection{Utilização}
Com os objetos necessários inicializados, é necessário ainda especificar qual (ou quais) a(s) \textit{SoundFont(s)} a utilizar. Para isso podemos chamar o método \texttt{Synth.LoadSoundFont} que recebe dois argumentos: uma \textit{string} com o caminho em disco do ficheiro \textit{SoundFont} a carregar, seguido dum booleano que indica se os \textit{presets} devem ser atualizados para os da nova \textit{SoundFont} (isto é, atribuir os instrumentos da \textit{SoundFont} aos canais automaticamente).

A função \texttt{Synth.NoteOn} recebe três argumentos: um inteiro a representar o canal, outro inteiro entre 0 e 127 a representar a nota, e finalmente outro inteiro também entre 0 e 127 a representar a velocidade da nota.

O canal (\textbf{channel}) representa qual o instrumento que vai reproduzir a nota em questão. Cada canal está atríbuido a um programa da SoundFont, e é possível a qualquer momento mudar o programa atribuido a qualquer canal através do método \texttt{Synth.ProgramChange}. Caso se tenha carregado mais do que uma \textit{SoundFont}, é possível usar o método \texttt{Synth.ProgramSelect}, que permite especificar o \texttt{id} da \textit{SoundFont} e do banco do instrumento a atribuir.

A chave (\textbf{key}) representa a nota a tocar. Sendo este valor um inteiro entre 0 e 127, é necessário saber como mapear as tradicionais notas músicais neste valor. Para isso, basta colocarmos as \textit{pich classes} e os seus respetivos acidentais \textit{sharp} numa lista ordenada (\verb|C, C#, D, D#, E, F, F#, G, G#, A, A#, B|) e associar a eles os inteiros entre 0 e 11 (inclusive). Depois apenas temos de somar a esse número a multiplicação da oitava da nota (a começar em 0) por 12. Podemos deste modo calcular, por exemplo, que a \textit{key} do C central (C4) é igual a 48 ($0 + 4 * 12$).
$$N + O * 12$$

A velocidade (\textbf{velocity}) é também um valor entre 0 e 127. Relacionando a velocidade com um piano físico, esta representa a força (ou velocidade) com que a tecla foi premida. Velocidades maiores geram sons mais altos, enquanto que velocidades mais baixas geram sons mais baixos, permitindo assim ao músico dar ou tirar enfase a uma nota relativamente às restantes. De notar que um valor igual a zero é o equivalente a invocar o método \texttt{Synth.NoteOff}.

A método \texttt{Synth.NoteOff}, por sua vez, recebe apenas dois argumentos (canal e chave), e deve ser chamada passsado algum tempo para terminar a nota. Podemos deste modo construir a analogia óbvia que o método \texttt{NoteOn} corresponde a uma tecla de piano ser premida, e \texttt{NoteOff} corresponde a essa tecla ser libertada.

\section{Gramáticas}
%% PEG vs CFG:
%%  - Em PEGs, o operador de escolha / é ordenado, i.e. é relevante a ordem por que as várias alternativas são colocadas.
%%    Numa CFG, o operador de escolha não é ordenado, o que pode resultar em gramáticas ambíguas, onde o mesmo input pode resultar em diferentes parse trees.
%% https://stackoverflow.com/questions/34976533/some-information-about-peg-needed
%% https://en.wikipedia.org/wiki/Parsing_expression_grammar
%% https://compilers.iecc.com/comparch/article/05-09-114


Para além dos aspetos técnicos da geração e reprodução de música já abordados neste relatório, existe também um componente fulcral relativo à análise e interpretação da linguagem que irá controlar a geração dos sons. Uma das primeiras decisões a ser tomada diz respeito à escolha do \textit{parser}, e possivelmente, do tipo de gramática que irá servir de base para a geração do mesmo.

Tradicionalmente, as gramáticas mais populares no campo de processamento de texto tendem a ser \acrfull{cfg}, que são usadas como \textit{input} nos geradores de \textit{parser} mais populares (Bison/YACC, ANTLR). Existem no entanto alternativas, algumas mais recentes, como as \acrfull{peg}, que trazem consigo diferenças que podem ser consideradas por alguns como vantagens ou desvantagens.

\subsection{Diferenças: \acrshort{cfg} vs \acrshort{peg}}

A diferença com maiores repercussões práticas entre as duas classes de gramáticas deve-se á semãntica atribuída ao operador de escolha, e a consequente \textbf{ambíguidade} (ou falta dela) na gramática. Nas gramáticas \acrshort{peg}, o operador é ordenado, o que significa que a ordem porque as alternativas aparecem é relevante durante o \textit{parse} do \textit{input}. Isto contrasta com a semântica nas \acrshort{cfg}, onde a ordem das alternativas é irrelevante. Isto pode no entanto levar a ambíguidades, onde o mesmo \textit{input}, descrito pela mesma gramática, pode resultar em duas árvores de \textit{parsing} diferentes. Isto é, as \acrshort{cfg} podem por essa razão ser ambíguas.

Tomemos como exemplo o famoso problema do \textit{dangling else}\cite{dangling-else} descrito nas duas classe de gramáticas:
\begin{center}
\begin{tabular}{c}
\begin{lstlisting}
 if (a) if (b) f1(); else f2();
\end{lstlisting}
\end{tabular}
\end{center}

\begin{lstlisting}[caption=Gramática]
 statement = ...
    | conditional_statement
 
 conditional_statement = ...
    | IF ( expression ) statement ELSE statement
    | IF ( expression ) statement
\end{lstlisting}

No caso de uma \acrshort{cfg}, sabendo que o operador de escolha \texttt{|} é comutativo, o seguinte \textit{input} será ambíguo, podendo resultar num \textit{if-else} dentro do \textit{if} ou num \textit{if} dentro de um \textit{if-else}.

Mas no caso de uma \acrshort{peg}, o resultado é claro: um \textit{if-else} dentro de um \textit{if}. Quando a primeira regra do condicional chega ao \texttt{statement}, este vai por sua vez chamar o não terminal \texttt{conditional\_statement}, que por sua vez irá consumir o \textit{input} até ao fim. Deste modo, quando a execução voltar ao primeiro \texttt{conditional\_statement}, esta irá falhar por não conseguir ler o \textit{else} (uma vez que já consumimos todo o texto de entrada). Deste modo irá usar a segunda alternativa, dando então o resultado previso.

Com este exemplo de \textit{backtracking} podemos também verificar um problema aparente nas gramáticas \acrshort{peg}. Falhando a primeira alternativa na produção \texttt{conditional\_statement}, a segunda irá ser testada. Mas é evidente, olhando para a gramática que a segunda alternativa é exatamente igual à parte inicial da primeira alternativa (que neste caso também corresponde á parte que teve sucesso). Em vez de voltar a testar as regras de uma forma \textit{naive}, as \acrlong{peg} guardam antes em \textit{cache} os resultados de testes anteriores, permitindo assim uma pesquisa em tempo linear relativamente ao tamanho do \textit{input}, à custa de uma maior utilização de memória.

%% TODO Talk about composition and why PEG grammars can do it better than CFG

\subsubsection{Resumo}
Em resumo, as três principais diferenças entre as tradicionais \acrfull{cfg} e as mais recentes \acrfull{peg} são:

\textbf{Ambiguidade}. O operador de escolha ser comutativo nas \acrshort{cfg} resulta em gramáticas que podem ser ambíguas para o mesmo \textit{input}. As \acrshort{peg} são determínisticas, mas exigem mais cuidado na ordem das produções, uma vez que tal afeta a semântica da gramática.

\textbf{Memoization} Para evitar \textit{backtracking} exponêncial, as \acrshort{peg} utilizam \textit{memoization} que lhes permite guardar em \textit{cache} resultados parciais durante o processo de \textit{parsing}. Isto reduz o tempo dispendido, pois evita fazer o \textit{parse} do mesmo texto pela mesma regra duas vezes. Mas também aumenta o consumo de memória, pois os resultados parciais têm de ser guardados até a análise terminar por completo.

\textbf{Composição} As \acrlong{peg} também têm a vantagem de oferecerem uma maior facilidade de composição. Em qualquer parte da gramática é possível trocar um terminal por um não terminal. Isto é, é extremamente fácil construir gramáticas mais modulares e compô-las entre si.

\subsection{Acompanhamentos Músicais}
A gramática de expressões ou acompanhamentos músicais tem como base fundamental os seguintes blocos: notas, pausas e modificadores. As notas são identificadas pelas letras A até G, seguindo a notação de \textit{Helmholtz}\cite{helmholtz-pitch-notation} para denotar as respetivas oitavas. Podem também ser seguidas de um número ou de uma fração, indicando a duração da nota.


\begin{center}
\begin{tabular}{c}
\textit{Exemplos de notas} \\
\begin{lstlisting}
C,, C, C c c' c'' c''' c'/4 A1/4 B2
\end{lstlisting}
\end{tabular}
\end{center}

As notas podem depois ser compostas sequencialmente (como demonstrado em cima) ou em paralelo (separados por uma barra vertical \texttt{|}). Devemos notar que o operador paralelo tem a menor precedência de todos, pelo que não é necessário agrupar as notas com parênteses quando se usa. Isto é, as duas expressões seguintes são equivalentes.

\begin{center}
\begin{tabular}{c}
\begin{lstlisting}
    A B C | D E F
( A B C ) | ( D E F )
\end{lstlisting}
\end{tabular}
\end{center}

É também possível agrupar estes blocos com recurso a parênteses. Os grupos herdam o contexto da expressão superior, mas as modificações ao seu contexto permanecem locais. Isto permite, por exemplo, modificar configurações para apenas um conjunto restrito de notas. No exemplo seguinte, a velocidade da nota \texttt{C} é 70, mas para o grupo de notas \texttt{A B} a velocidade é 127.

\begin{center}
\begin{tabular}{c}
\begin{lstlisting}
 v70 (v127 A B) C
\end{lstlisting}
\end{tabular}
\end{center}

Os modificadores disponíveis são:
\begin{description}
 \item[Velocity] A velocidade das notas, tendo o formato \texttt{[vV][0-9]+}.
 \item[Duração] A duração das notas, tendo o formato \texttt{[lL][0-9]+} ou \\ \texttt{[lL][0-9]+/[0-9]+}.
 \item[Tempo] O número de batidas por minuto (BPM) que definem a velocidade a que as notas são tocadas, tendo o formato \texttt{[tT][0-9]+}.
 \item[Assinatura de Tempo] Define a assinatura de tempo, que define o tipo de batida da música e o comprimento de uma barra na pauta musical. Tem o formato \texttt{[sS][0-9]+/[0-9]}.
\end{description}

É também poossível definir qual o instrumento a ser utilizado para as notas. Todas as notas pertencentes ao mesmo contexto depois do modificador utilizarão esse instrumento.

\begin{center}
\begin{tabular}{c}
\begin{lstlisting} 
 (:cello A F | :violin A D)
\end{lstlisting}
\end{tabular}
\end{center}

Para além destas funcionalidades, também existe algum açúcar sintático para algumas das tarefas mais comuns na construção de acompanhamentos, como tocar acordes ou repetir padrões.

\begin{center}
\begin{tabular}{c}
\begin{lstlisting}
( [BG]*2 [B2G2] )*3
\end{lstlisting}
\end{tabular}
\end{center}

\begin{center}
\begin{tabular}{c}
\begin{lstlisting}

\end{lstlisting}
\end{tabular}
\end{center}

A gramática completa pode ser analizada no \textbf{Anexo \ref{appendix:grammar}}.

\bibliographystyle{unsrt}
\bibliography{bibliography}

\newpage
\begin{appendices}
\section{Gramática} 
\label{appendix:grammar}
\begin{lstlisting}
body = _ expression _

expression = parallel

parallel
    = sequence _ "|" _ parallel
    | sequence

sequence <MusicNode>
    = repeat _ sequence
    | repeat
    
repeat
    = expressionUnambiguous _ "*" _ integer
    | expressionUnambiguous

expressionUnambiguous
    = group | chord | note | rest | modifier | instrumentModifier

group
    = "(" _ expression _ ")"

note
    = notePitch _ noteValue
    | notePitch

chord = "[" _ chordBody  _ "]"

chordBody
    = note _ chordBody
    | note

rest
    = "r" _ noteValue
    | "r"

noteValue
    = "/" _ integer
    | integer _ "/" _ integer
    | integer

notePitch
    = [cdefgab] "'"*
    | [CDEFGAB] ","*

modifier
    = [tT] _ integer
    | [vV] _ integer
    | [lL] _ noteValue
    | [sS] _ integer _ "/" _ integer
    | [sS] _ integer
    | [oO] _ integer

instrumentModifier
    = ":" alphanumeric _ sequence
    
integer
    = [0-9]+

alphanumeric
    = [a-zA-Z][a-zA-Z0-9]*

_ = [ \t\r\n]*
\end{lstlisting}
\end{appendices}


\end{document}
